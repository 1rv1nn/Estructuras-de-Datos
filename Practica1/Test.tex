\documentclass[a4paper,10pt]{article} 
\usepackage[top=2cm,bottom=2cm,left=2cm,rigth=2cm,heightrounded]{geometry}
\usepackage[utf8]{inputenc}
\usepackage{graphicx}
\usepackage{multirow} 
\usepackage[spanish]{babel}
\usepackage[usenames]{color}
\usepackage{dsfont}
\usepackage{amssymb}
\usepackage{amsmath}
\usepackage{bbding}  
\usepackage[dvipsnames]{xcolor}
\usepackage{csquotes}
\usepackage[export]{adjustbox}
\usepackage[all]{nowidow} 
\usepackage{csquotes} 
\everymath{\displaystyle}
\usepackage{setspace}
\usepackage[ddmmyyyy]{datetime} 
\renewcommand{\dateseparator}{-} 
\usepackage{fancyhdr}
\usepackage{amsmath,xcolor}
\usepackage[inline]{enumitem}
\usepackage[table]{xcolor}
\usepackage{hhline,colortbl}
\usepackage{tikz,lipsum,lmodern}
\usepackage[most]{tcolorbox}

%%%%%AQUÍ SE ENCUENTRAN LOS COLORES :)%%%%%%%%
\definecolor{babypink}{rgb}{0.96, 0.76, 0.76}
\definecolor{celadon}{rgb}{0.67, 0.88, 0.69}
\definecolor{sangria}{rgb}{0.57, 0.0, 0.04}
\definecolor{green(ncs)}{rgb}{0.0, 0.62, 0.42}
\pagecolor{black}
\color{white}

\newtcolorbox{mybox}{colback=black!20!black,colframe=cyan!85!white}

\pagestyle{fancy} 
\fancyhead{}\renewcommand{\headrulewidth}{0pt} 
\fancyfoot[C]{} 
\fancyfoot[R]{\thepage} 
\newcommand{\note}[1]{\marginpar{\scriptsize \textcolor{red}{#1}}} 
\begin{document}
\fancyhead[C]{}
\begin{minipage}{0.295\textwidth} 
\raggedright
Equipo\\    
\footnotesize 
\colorbox[rgb]{0.67, 0.88, 0.69}{\textcolor{black}{Arollo Martínez Erick Daniel}}
\\\colorbox[rgb]{0.96, 0.76, 0.76}{\textcolor{black}{Cruz González Irvin Javier}}
\textcolor[rgb]{0.0, 0.72, 0.92}{\medskip\hrule}
\end{minipage}
\begin{minipage}{0.4\textwidth} 
\centering 
\large 
\textbf{Estructuras de Datos}\\ 
\normalsize 
Complejidad Computacional\\Practica 01\\
\end{minipage}
\begin{minipage}{0.295\textwidth} 
\raggedleft
\today\\ 
\footnotesize
erickarrollo@ciencias.unam.mx
1rv1n@ciencias.unam.mx
\textcolor[rgb]{0.0, 0.72, 0.92}{\medskip\hrule}
\end{minipage}

\begin{enumerate}
    
    \item[2.4] \textbf{Actividad 4}
    \arrayrulecolor{cyan}
    \begin{table}[htb]
        \centering
        \begin{tabular}{|l|l|l|}
        \hline
        \multicolumn{3}{|c|}{findFirstAndLast} \\ \hline
        Entradas & Nanosegundos algoritmo 1 & Nanosegundos algoritmo 2 \\
        \hline \hline
        [1,4,2,1,6,2,9], 2 & \hspace{1.3cm}{$\approx 358$} & \hspace{1.3cm}{$\approx 202$} \\ \hline
        [4,2,7,5,4,3,7,2,5,3,4,1], 15 &  \hspace{1.3cm}{$\approx 329$}& \hspace{1.3cm}{$\approx 197$} \\ \hline
        [3,2,1,4,2], 1 & \hspace{1.3cm}{$\approx 280$} & \hspace{1.3cm}{$\approx 307$} \\ \hline
        \end{tabular}
        \end{table}

        \begin{tcolorbox}[colback=black!20!black,colframe=cyan!85!black]
           \textcolor{white}{ \textbf{Justificación.\\}El algoritmo 2 que se implemento consta de un solo for anidando if-else por lo que esta última estructura de control tiene una complejidad $O(1)$,
                             disminuyendo considerablemente la complejidad que se tenía al inicio,y aproximandose a $O(n/2)$.}
          \end{tcolorbox}
        
       
        \arrayrulecolor{sangria}   
          \begin{table}[htb]
            \centering
            \begin{tabular}{|l|l|l|}
            \hline
            \multicolumn{3}{|c|}{isSudokuValid} \\ \hline
            Entradas & Milisegundos algoritmo 1 & Milisegundos algoritmo 2 \\
            \hline \hline
            ejemplo2a & \hspace{1.3cm}{$\approx 38$} & \hspace{1.3cm}{$\approx 9$} \\ \hline
            ejemplo2b & \hspace{1.3cm}{$\approx 33$} & \hspace{1.3cm}{$\approx 8$} \\ \hline
            \end{tabular}
            \end{table}

            \begin{tcolorbox}[colback=black!20!black,colframe=sangria!85!black]
                \textcolor{white}{ \textbf{Justificación.\\}EL algoritmo 2 que se implemento consta de métodos auxiliares utilizando un solo for con if-else por lo que 
                                    sus complejidades son $O(n)$ y al implementar el método isSudokuValid se sumaron y multiplicaron estas dichas complejidades,generando así una complejidad $O(n^{2})$ }
               \end{tcolorbox} 
               
               
               \arrayrulecolor{green(ncs)}     
               \begin{table}[htb]
                \centering
                \begin{tabular}{|l|l|l|}
                \hline
                \multicolumn{3}{|c|}{rotateArray} \\ \hline
                Entradas & Milisegundos algoritmo 1 & Milisegundos algoritmo 2 \\
                \hline \hline
                [1,4,2,1,6,2,9], 5 & \hspace{1.3cm}{$\approx 37$}  & \hspace{1.3cm}{$\approx 6$} \\ \hline
                [4,2,7,5,4,3,7,2,5,3,4,1], 0 & \hspace{1.3cm}{$\approx 37$} & \hspace{1.3cm}{$\approx 1$} \\ \hline
                [3,2,1,4,2], 2 & \hspace{1.3cm}{$\approx 37$} & \hspace{1.3cm}{$\approx 1$} \\ \hline
                \end{tabular}
                \end{table}  
                
                \begin{tcolorbox}[colback=black!20!black,colframe=green(ncs)!85!black]
                    \textcolor{white}{ \textbf{Justificación.\\}El algoritmo 2 se implemento a base de de if-else y un for teniendo una complejidad $O(1)$ y 
                                      $O(n)$ respectivamente, dando como resultado la complejidad requerida $O(n)$.}
                   \end{tcolorbox} 
                       
                 

\end{enumerate}
\end{document}   

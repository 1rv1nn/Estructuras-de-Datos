\documentclass[a4paper,10pt]{article} 
\usepackage[top=1cm,bottom=2cm,left=1cm,rigth=1cm,heightrounded]{geometry}
\usepackage[utf8]{inputenc}
\usepackage{graphicx}
\usepackage{multirow} 
\usepackage[spanish]{babel}
\usepackage[usenames]{color}
\usepackage{dsfont}
\usepackage{amssymb}
\usepackage{amsmath}
\usepackage{bbding}  
\usepackage[dvipsnames]{xcolor}
\usepackage{csquotes}
\usepackage[export]{adjustbox}
\usepackage[all]{nowidow} 
\usepackage{csquotes} 
\everymath{\displaystyle}
\usepackage{setspace}
\usepackage[yyyymmdd]{datetime} 
\renewcommand{\dateseparator}{-} 
\usepackage{fancyhdr}
\usepackage{background}
\usepackage{lipsum}
\usepackage{tcolorbox}
\usepackage{listings}
\usepackage{minted}


%colors
\definecolor{antiquebrass}{rgb}{0.8, 0.58, 0.46}
\definecolor{babypink}{rgb}{0.96, 0.76, 0.76}
\definecolor{buff}{rgb}{0.94, 0.86, 0.51}

\newlength\mylen
\setlength\mylen{\dimexpr\paperwidth/38\relax}

\SetBgScale{1}
\SetBgAngle{0}
\SetBgColor{black!13}
\SetBgContents{\tikz{\draw[step=\mylen] (-.5\paperwidth,-.6\paperheight) grid (.9\paperwidth,2.98\paperheight);}}

\pagestyle{fancy} 
\fancyhead{}\renewcommand{\headrulewidth}{0pt} 
\fancyfoot[C]{} 
\fancyfoot[R]{\thepage} 
\newcommand{\note}[1]{\marginpar{\scriptsize \textcolor{red}{#1}}} 
\begin{document}
\fancyhead[C]{}
\begin{minipage}{0.295\textwidth} 
\raggedright
Equipo\\    
\footnotesize 
\colorbox[rgb]{0.94, 0.86, 0.51}{\textcolor{black}{Arollo Martínez Erick Daniel}}
\\\colorbox[rgb]{0.96, 0.76, 0.76}{\textcolor{black}{Cruz González Irvin Javier}}
\textcolor[rgb]{0.8, 0.58, 0.46}{\medskip\hrule}
\end{minipage}
\begin{minipage}{0.4\textwidth} 
\centering 
\large 
\textbf{Estructuras de Datos }\\ 
\normalsize 
Complejidad Computacional\\
Tarea 1
\end{minipage}
\begin{minipage}{0.295\textwidth} 
\raggedleft
\today\\ 
\footnotesize
erickarrollo@ciencias.unam.mx
1rv1n@ciencias.unam.mx
\textcolor[rgb]{0.8, 0.58, 0.46}{\medskip\hrule}
\end{minipage}

\begin{enumerate}
    \item \textbf{Ejercicio 1}
    
     Calcula el tiempo de ejecución en el peor de los casos para los siguientes  métodos :
            
    \textbf{Problema 1}
    %problema 1
    \begin{lstlisting}[frame = single] 
problema1(A){
    suma = 0;
    for(posicion = 1; i<= n; i++){  ->>>> n
        suma = suma + A[posicion];  ->>>> 1+1+1+1=4 
    }//end for
    return suma;            ->>>>>4n
}
\end{lstlisting}

    %solución
    \textbf{\textcolor{red}{Solución:}} Entra al for con $n$ iteraciones,
   
    %problema2
    \textbf{Problema 2}
    \begin{lstlisting}[frame = single]
public static int problema2(int n){
    if(n != 0){
        int x = n + 3;
        int y = n + x + y;
        return y;
    }else{
        return 10;
    }
}
\end{lstlisting}

    %problema3
    \textbf{Problema 3}
    \begin{lstlisting}[frame = single]
public static int problema3(int n){
    int x = 0;
    for(int i = 0; i < n; i ++){
       for(int j = 1; j < n; j++){
           x++;
       }
    }
    return x;
}
\end{lstlisting}


    \item \textbf{Ejercicio 2}

    Calcula el tiempo de ejecución en el peor de los casos para los siguientes métodos y determina su complejidad.  

    %problema4
    \textbf{Problema 4}
    \begin{lstlisting}[frame = single]
/*
 * n es un entero postivo que ademas es potencia de 2
 **/
public static int problema4(int n){
    int i = n;
    int contador = 0;
    while(i > 1){
        i = i / 2;
        contador++;
    }
    return contador
}
\end{lstlisting}

\newpage

%problema5
 \textbf{Problema 5}
\begin{lstlisting}[frame = single]
public static int problema5(int t){
int suma = 0;
for(int i = 0; i < t; i++){
    suma += problema5(i);
}
return suma;
}
\end{lstlisting}

%problema6
 \textbf{Problema 6}
\begin{lstlisting}[frame = single]
public int problema6(int n){
int suma = 0;
for(int i = 0; i < n; i++){
    for(int j = n -1; j >=0 ; j++){
        suma = suma  + problema6;
    }
}
return suma;
}
\end{lstlisting}

\item \textbf{Ejercicio 3}

        \textbf{Definición:} Sean $f(n)$ y $g(n)$ funciones de complejidad. Decimos que $f(n)$ es $O-grande$ de $g(n)$ y $g(n)$ representa una cota asintótica superior para $f(n)$ si $\exists \: c \in \mathds{R^+}$  y $\exists \: n_0 \in \mathds{N}\cup \{0\}$ tales que $\forall _n\geq n_0 : 0 \leq f(n) \leq c \cdot g(n)$.\\

        Demuestra cada uno de los siguientes ejercicios:
        %problema7,8,9,10
\begin{itemize}
    \item Sea $T(n) = 5\sqrt{n} + 6n^2 $, P.D que $T(n) = 5\sqrt{n} + 6n^2 \in O(n^2) $
    \item Sea $T(n) = 83 n^2 + 31 $, P.D que $T(n) = 83 n^2 + 31  \in O(n^2) $
    \item Sea $T(n) = 53n + 3 log(n) $, P.D que $T(n) = 53n + 3 log(n)   \in O(n) $
     \item Sea $T(n) = 37n^3 + n^2log(n) + 37 $, P.D que $T(n) =  37n^3 + n^2log(n) + 37  \in O(n^3) $
    
\end{itemize}








\end{enumerate}
\end{document}